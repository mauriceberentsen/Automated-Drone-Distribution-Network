\hypertarget{GettingStarted_HandlingData_8ino-example}{}\section{Getting\+Started\+\_\+\+Handling\+Data.\+ino}
{\bfseries Dec 2014 -\/ T\+M\+Rh20}~\newline


This example demonstrates how to send multiple variables in a single payload and work with data. As usual, it is generally important to include an incrementing value like \hyperlink{group__Porting__General_gad5b3ec1ce839fa1c4337a7d0312e9749}{millis()} in the payloads to prevent errors.


\begin{DoxyCodeInclude}

\textcolor{comment}{/*}
\textcolor{comment}{* Getting Started example sketch for nRF24L01+ radios}
\textcolor{comment}{* This is an example of how to send data from one node to another using data structures}
\textcolor{comment}{* Updated: Dec 2014 by TMRh20}
\textcolor{comment}{*/}

\textcolor{preprocessor}{#include <SPI.h>}
\textcolor{preprocessor}{#include "\hyperlink{RF24_8h}{RF24.h}"}

byte addresses[][6] = \{\textcolor{stringliteral}{"1Node"},\textcolor{stringliteral}{"2Node"}\};


\textcolor{comment}{/****************** User Config ***************************/}
\textcolor{comment}{/***      Set this radio as radio number 0 or 1         ***/}
\textcolor{keywordtype}{bool} radioNumber = 1;

\textcolor{comment}{/* Hardware configuration: Set up nRF24L01 radio on SPI bus plus pins 7 & 8 */}
\hyperlink{classRF24}{RF24} radio(7,8);
\textcolor{comment}{/**********************************************************/}


\textcolor{comment}{// Used to control whether this node is sending or receiving}
\textcolor{keywordtype}{bool} role = 0;

\textcolor{keyword}{struct }dataStruct\{
  \textcolor{keywordtype}{unsigned} \textcolor{keywordtype}{long} \_micros;
  \textcolor{keywordtype}{float} value;
\}myData;

\textcolor{keywordtype}{void} setup() \{

  Serial.begin(115200);
  Serial.println(F(\textcolor{stringliteral}{"RF24/examples/GettingStarted\_HandlingData"}));
  Serial.println(F(\textcolor{stringliteral}{"*** PRESS 'T' to begin transmitting to the other node"}));
  
  radio.\hyperlink{classRF24_a048a20c73c7d9b2e02dcbae6fb9c4ba8}{begin}();

  \textcolor{comment}{// Set the PA Level low to prevent power supply related issues since this is a}
 \textcolor{comment}{// getting\_started sketch, and the likelihood of close proximity of the devices. RF24\_PA\_MAX is default.}
  radio.\hyperlink{classRF24_adedac579590a668ae97baccab284de8a}{setPALevel}(\hyperlink{RF24_8h_a1e4cd0bea93e6b43422855fb0120aacea7d8d09f4a047b7c22655e56c98ca010c}{RF24\_PA\_LOW});
  
  \textcolor{comment}{// Open a writing and reading pipe on each radio, with opposite addresses}
  \textcolor{keywordflow}{if}(radioNumber)\{
    radio.\hyperlink{classRF24_af2e409e62d49a23e372a70b904ae30e1}{openWritingPipe}(addresses[1]);
    radio.\hyperlink{classRF24_a9edc910ccc1ffcff56814b08faca5535}{openReadingPipe}(1,addresses[0]);
  \}\textcolor{keywordflow}{else}\{
    radio.\hyperlink{classRF24_af2e409e62d49a23e372a70b904ae30e1}{openWritingPipe}(addresses[0]);
    radio.\hyperlink{classRF24_a9edc910ccc1ffcff56814b08faca5535}{openReadingPipe}(1,addresses[1]);
  \}
  
  myData.value = 1.22;
  \textcolor{comment}{// Start the radio listening for data}
  radio.\hyperlink{classRF24_a30a2733a3889bdc331fe2d2f4f0f7b39}{startListening}();
\}




\textcolor{keywordtype}{void} loop() \{
  
  
\textcolor{comment}{/****************** Ping Out Role ***************************/}  
\textcolor{keywordflow}{if} (role == 1)  \{
    
    radio.\hyperlink{classRF24_a6f144d73fc447c8ac2d1a4166210fd88}{stopListening}();                                    \textcolor{comment}{// First, stop listening so we
       can talk.}
    
    
    Serial.println(F(\textcolor{stringliteral}{"Now sending"}));

    myData.\_micros = micros();
     \textcolor{keywordflow}{if} (!radio.\hyperlink{classRF24_a4cd4c198a47704db20b6b5cf0731cd58}{write}( &myData, \textcolor{keyword}{sizeof}(myData) ))\{
       Serial.println(F(\textcolor{stringliteral}{"failed"}));
     \}
        
    radio.\hyperlink{classRF24_a30a2733a3889bdc331fe2d2f4f0f7b39}{startListening}();                                    \textcolor{comment}{// Now, continue listening}
    
    \textcolor{keywordtype}{unsigned} \textcolor{keywordtype}{long} started\_waiting\_at = micros();               \textcolor{comment}{// Set up a timeout period, get the current
       microseconds}
    \textcolor{keywordtype}{boolean} timeout = \textcolor{keyword}{false};                                   \textcolor{comment}{// Set up a variable to indicate if a
       response was received or not}
    
    \textcolor{keywordflow}{while} ( ! radio.\hyperlink{classRF24_a127105eb7a3b351cfe777c1cec50627a}{available}() )\{                             \textcolor{comment}{// While nothing is received}
      \textcolor{keywordflow}{if} (micros() - started\_waiting\_at > 200000 )\{            \textcolor{comment}{// If waited longer than 200ms, indicate
       timeout and exit while loop}
          timeout = \textcolor{keyword}{true};
          \textcolor{keywordflow}{break};
      \}      
    \}
        
    \textcolor{keywordflow}{if} ( timeout )\{                                             \textcolor{comment}{// Describe the results}
        Serial.println(F(\textcolor{stringliteral}{"Failed, response timed out."}));
    \}\textcolor{keywordflow}{else}\{
                                                                \textcolor{comment}{// Grab the response, compare, and send to
       debugging spew}
        radio.\hyperlink{classRF24_a8e2eacacfba96426c192066f04054c5b}{read}( &myData, \textcolor{keyword}{sizeof}(myData) );
        \textcolor{keywordtype}{unsigned} \textcolor{keywordtype}{long} time = micros();
        
        \textcolor{comment}{// Spew it}
        Serial.print(F(\textcolor{stringliteral}{"Sent "}));
        Serial.print(time);
        Serial.print(F(\textcolor{stringliteral}{", Got response "}));
        Serial.print(myData.\_micros);
        Serial.print(F(\textcolor{stringliteral}{", Round-trip delay "}));
        Serial.print(time-myData.\_micros);
        Serial.print(F(\textcolor{stringliteral}{" microseconds Value "}));
        Serial.println(myData.value);
    \}

    \textcolor{comment}{// Try again 1s later}
    \hyperlink{group__Porting__General_ga70a331e8ddf9acf9d33c47b71cda4c5f}{delay}(1000);
  \}



\textcolor{comment}{/****************** Pong Back Role ***************************/}

  \textcolor{keywordflow}{if} ( role == 0 )
  \{
    
    \textcolor{keywordflow}{if}( radio.\hyperlink{classRF24_a127105eb7a3b351cfe777c1cec50627a}{available}())\{
                                                           \textcolor{comment}{// Variable for the received timestamp}
      \textcolor{keywordflow}{while} (radio.\hyperlink{classRF24_a127105eb7a3b351cfe777c1cec50627a}{available}()) \{                          \textcolor{comment}{// While there is data ready}
        radio.\hyperlink{classRF24_a8e2eacacfba96426c192066f04054c5b}{read}( &myData, \textcolor{keyword}{sizeof}(myData) );             \textcolor{comment}{// Get the payload}
      \}
     
      radio.\hyperlink{classRF24_a6f144d73fc447c8ac2d1a4166210fd88}{stopListening}();                               \textcolor{comment}{// First, stop listening so we can
       talk  }
      myData.value += 0.01;                                \textcolor{comment}{// Increment the float value}
      radio.\hyperlink{classRF24_a4cd4c198a47704db20b6b5cf0731cd58}{write}( &myData, \textcolor{keyword}{sizeof}(myData) );              \textcolor{comment}{// Send the final one back.      }
      radio.\hyperlink{classRF24_a30a2733a3889bdc331fe2d2f4f0f7b39}{startListening}();                              \textcolor{comment}{// Now, resume listening so we
       catch the next packets.     }
      Serial.print(F(\textcolor{stringliteral}{"Sent response "}));
      Serial.print(myData.\_micros);  
      Serial.print(F(\textcolor{stringliteral}{" : "}));
      Serial.println(myData.value);
   \}
 \}




\textcolor{comment}{/****************** Change Roles via Serial Commands ***************************/}

  \textcolor{keywordflow}{if} ( Serial.available() )
  \{
    \textcolor{keywordtype}{char} c = toupper(Serial.read());
    \textcolor{keywordflow}{if} ( c == \textcolor{charliteral}{'T'} && role == 0 )\{      
      Serial.print(F(\textcolor{stringliteral}{"*** CHANGING TO TRANSMIT ROLE -- PRESS 'R' TO SWITCH BACK"}));
      role = 1;                  \textcolor{comment}{// Become the primary transmitter (ping out)}
    
   \}\textcolor{keywordflow}{else}
    \textcolor{keywordflow}{if} ( c == \textcolor{charliteral}{'R'} && role == 1 )\{
      Serial.println(F(\textcolor{stringliteral}{"*** CHANGING TO RECEIVE ROLE -- PRESS 'T' TO SWITCH BACK"}));      
       role = 0;                \textcolor{comment}{// Become the primary receiver (pong back)}
       radio.\hyperlink{classRF24_a30a2733a3889bdc331fe2d2f4f0f7b39}{startListening}();
       
    \}
  \}


\} \textcolor{comment}{// Loop}
\end{DoxyCodeInclude}
 