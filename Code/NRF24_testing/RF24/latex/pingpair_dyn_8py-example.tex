\hypertarget{pingpair_dyn_8py-example}{}\section{pingpair\+\_\+dyn.\+py}
This is a python example for R\+Pi of how to use payloads of a varying (dynamic) size.


\begin{DoxyCodeInclude}
1 \textcolor{comment}{#!/usr/bin/env python}
2 
3 \textcolor{comment}{#}
4 \textcolor{comment}{# Example using Dynamic Payloads}
5 \textcolor{comment}{# }
6 \textcolor{comment}{#  This is an example of how to use payloads of a varying (dynamic) size.}
7 \textcolor{comment}{# }
8 
9 \textcolor{keyword}{from} \_\_future\_\_ \textcolor{keyword}{import} print\_function
10 \textcolor{keyword}{import} time
11 \textcolor{keyword}{from} RF24 \textcolor{keyword}{import} *
12 \textcolor{keyword}{import} RPi.GPIO \textcolor{keyword}{as} GPIO
13 
14 irq\_gpio\_pin = \textcolor{keywordtype}{None}
15 
16 
18 
19 \textcolor{comment}{# CE Pin, CSN Pin, SPI Speed}
20 
21 \textcolor{comment}{# Setup for GPIO 22 CE and CE0 CSN with SPI Speed @ 8Mhz}
22 \textcolor{comment}{#radio = RF24(RPI\_V2\_GPIO\_P1\_15, BCM2835\_SPI\_CS0, BCM2835\_SPI\_SPEED\_8MHZ)}
23 
24 \textcolor{comment}{#RPi B}
25 \textcolor{comment}{# Setup for GPIO 15 CE and CE1 CSN with SPI Speed @ 8Mhz}
26 \textcolor{comment}{#radio = RF24(RPI\_V2\_GPIO\_P1\_15, BCM2835\_SPI\_CS0, BCM2835\_SPI\_SPEED\_8MHZ)}
27 
28 \textcolor{comment}{#RPi B+}
29 \textcolor{comment}{# Setup for GPIO 22 CE and CE0 CSN for RPi B+ with SPI Speed @ 8Mhz}
30 \textcolor{comment}{#radio = RF24(RPI\_BPLUS\_GPIO\_J8\_15, RPI\_BPLUS\_GPIO\_J8\_24, BCM2835\_SPI\_SPEED\_8MHZ)}
31 
32 \textcolor{comment}{# RPi Alternate, with SPIDEV - Note: Edit RF24/arch/BBB/spi.cpp and  set 'this->device =
       "/dev/spidev0.0";;' or as listed in /dev}
33 radio = \hyperlink{classRF24}{RF24}(22, 0);
34 
35 
36 \textcolor{comment}{# Setup for connected IRQ pin, GPIO 24 on RPi B+; uncomment to activate}
37 \textcolor{comment}{#irq\_gpio\_pin = RPI\_BPLUS\_GPIO\_J8\_18}
38 \textcolor{comment}{#irq\_gpio\_pin = 24}
39 
40 
41 \textcolor{keyword}{def }try\_read\_data(channel=0):
42     \textcolor{keywordflow}{if} radio.available():
43         \textcolor{keywordflow}{while} radio.available():
44             len = radio.getDynamicPayloadSize()
45             receive\_payload = radio.read(len)
46             print(\textcolor{stringliteral}{'Got payload size=\{\} value="\{\}"'}.format(len, receive\_payload.decode(\textcolor{stringliteral}{'utf-8'})))
47             \textcolor{comment}{# First, stop listening so we can talk}
48             radio.stopListening()
49 
50             \textcolor{comment}{# Send the final one back.}
51             radio.write(receive\_payload)
52             print(\textcolor{stringliteral}{'Sent response.'})
53 
54             \textcolor{comment}{# Now, resume listening so we catch the next packets.}
55             radio.startListening()
56 
57 pipes = [0xF0F0F0F0E1, 0xF0F0F0F0D2]
58 min\_payload\_size = 4
59 max\_payload\_size = 32
60 payload\_size\_increments\_by = 1
61 next\_payload\_size = min\_payload\_size
62 inp\_role = \textcolor{stringliteral}{'none'}
63 send\_payload = b\textcolor{stringliteral}{'ABCDEFGHIJKLMNOPQRSTUVWXYZ789012'}
64 millis = \textcolor{keyword}{lambda}: int(round(time.time() * 1000))
65 
66 print(\textcolor{stringliteral}{'pyRF24/examples/pingpair\_dyn/'})
67 radio.begin()
68 radio.enableDynamicPayloads()
69 radio.setRetries(5,15)
70 radio.printDetails()
71 
72 print(\textcolor{stringliteral}{' ************ Role Setup *********** '})
73 \textcolor{keywordflow}{while} (inp\_role !=\textcolor{stringliteral}{'0'}) \textcolor{keywordflow}{and} (inp\_role !=\textcolor{stringliteral}{'1'}):
74     inp\_role = str(input(\textcolor{stringliteral}{'Choose a role: Enter 0 for receiver, 1 for transmitter (CTRL+C to exit) '}))
75 
76 \textcolor{keywordflow}{if} inp\_role == \textcolor{stringliteral}{'0'}:
77     print(\textcolor{stringliteral}{'Role: Pong Back, awaiting transmission'})
78     \textcolor{keywordflow}{if} irq\_gpio\_pin \textcolor{keywordflow}{is} \textcolor{keywordflow}{not} \textcolor{keywordtype}{None}:
79         \textcolor{comment}{# set up callback for irq pin}
80         GPIO.setmode(GPIO.BCM)
81         GPIO.setup(irq\_gpio\_pin, GPIO.IN, pull\_up\_down=GPIO.PUD\_UP)
82         GPIO.add\_event\_detect(irq\_gpio\_pin, GPIO.FALLING, callback=try\_read\_data)
83 
84     radio.openWritingPipe(pipes[1])
85     radio.openReadingPipe(1,pipes[0])
86     radio.startListening()
87 \textcolor{keywordflow}{else}:
88     print(\textcolor{stringliteral}{'Role: Ping Out, starting transmission'})
89     radio.openWritingPipe(pipes[0])
90     radio.openReadingPipe(1,pipes[1])
91 
92 \textcolor{comment}{# forever loop}
93 \textcolor{keywordflow}{while} 1:
94     \textcolor{keywordflow}{if} inp\_role == \textcolor{stringliteral}{'1'}:   \textcolor{comment}{# ping out}
95         \textcolor{comment}{# The payload will always be the same, what will change is how much of it we send.}
96 
97         \textcolor{comment}{# First, stop listening so we can talk.}
98         radio.stopListening()
99 
100         \textcolor{comment}{# Take the time, and send it.  This will block until complete}
101         print(\textcolor{stringliteral}{'Now sending length \{\} ... '}.format(next\_payload\_size), end=\textcolor{stringliteral}{""})
102         radio.write(send\_payload[:next\_payload\_size])
103 
104         \textcolor{comment}{# Now, continue listening}
105         radio.startListening()
106 
107         \textcolor{comment}{# Wait here until we get a response, or timeout}
108         started\_waiting\_at = \hyperlink{group__Porting__General_gad5b3ec1ce839fa1c4337a7d0312e9749}{millis}()
109         timeout = \textcolor{keyword}{False}
110         \textcolor{keywordflow}{while} (\textcolor{keywordflow}{not} radio.available()) \textcolor{keywordflow}{and} (\textcolor{keywordflow}{not} timeout):
111             \textcolor{keywordflow}{if} (\hyperlink{group__Porting__General_gad5b3ec1ce839fa1c4337a7d0312e9749}{millis}() - started\_waiting\_at) > 500:
112                 timeout = \textcolor{keyword}{True}
113 
114         \textcolor{comment}{# Describe the results}
115         \textcolor{keywordflow}{if} timeout:
116             print(\textcolor{stringliteral}{'failed, response timed out.'})
117         \textcolor{keywordflow}{else}:
118             \textcolor{comment}{# Grab the response, compare, and send to debugging spew}
119             len = radio.getDynamicPayloadSize()
120             receive\_payload = radio.read(len)
121 
122             \textcolor{comment}{# Spew it}
123             print(\textcolor{stringliteral}{'got response size=\{\} value="\{\}"'}.format(len, receive\_payload.decode(\textcolor{stringliteral}{'utf-8'})))
124 
125         \textcolor{comment}{# Update size for next time.}
126         next\_payload\_size += payload\_size\_increments\_by
127         \textcolor{keywordflow}{if} next\_payload\_size > max\_payload\_size:
128             next\_payload\_size = min\_payload\_size
129         time.sleep(0.1)
130     \textcolor{keywordflow}{else}:
131         \textcolor{comment}{# Pong back role.  Receive each packet, dump it out, and send it back}
132 
133         \textcolor{comment}{# if there is data ready}
134         \textcolor{keywordflow}{if} irq\_gpio\_pin \textcolor{keywordflow}{is} \textcolor{keywordtype}{None}:
135             \textcolor{comment}{# no irq pin is set up -> poll it}
136             try\_read\_data()
137         \textcolor{keywordflow}{else}:
138             \textcolor{comment}{# callback routine set for irq pin takes care for reading -}
139             \textcolor{comment}{# do nothing, just sleeps in order not to burn cpu by looping}
140             time.sleep(1000)
141 
\end{DoxyCodeInclude}
 