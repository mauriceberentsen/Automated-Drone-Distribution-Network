\hypertarget{transfer_8cpp-example}{}\section{transfer.\+cpp}
{\bfseries For Linux}~\newline
 This example demonstrates half-\/rate transfer using the F\+I\+FO buffers~\newline


It is an example of how to use the \hyperlink{classRF24}{R\+F24} class. Write this sketch to two different nodes. Put one of the nodes into \textquotesingle{}transmit\textquotesingle{} mode by connecting ~\newline
 with the serial monitor and sending a \textquotesingle{}T\textquotesingle{}. The data transfer will begin, with the receiver displaying the payload count. (32\+Byte Payloads) ~\newline



\begin{DoxyCodeInclude}
\textcolor{comment}{/*}
\textcolor{comment}{TMRh20 2014}
\textcolor{comment}{}
\textcolor{comment}{ This program is free software; you can redistribute it and/or}
\textcolor{comment}{ modify it under the terms of the GNU General Public License}
\textcolor{comment}{ version 2 as published by the Free Software Foundation.}
\textcolor{comment}{ */}

\textcolor{preprocessor}{#include <cstdlib>}
\textcolor{preprocessor}{#include <iostream>}
\textcolor{preprocessor}{#include <sstream>}
\textcolor{preprocessor}{#include <string>}
\textcolor{preprocessor}{#include <\hyperlink{RF24_8h}{RF24/RF24.h}>}
\textcolor{preprocessor}{#include <unistd.h>}

\textcolor{keyword}{using namespace }\hyperlink{namespacestd}{std};
\textcolor{comment}{//}
\textcolor{comment}{// Hardware configuration}
\textcolor{comment}{//}

\textcolor{comment}{/****************** Raspberry Pi ***********************/}

\textcolor{comment}{// Radio CE Pin, CSN Pin, SPI Speed}
\textcolor{comment}{// See http://www.airspayce.com/mikem/bcm2835/group\_\_constants.html#ga63c029bd6500167152db4e57736d0939 and
       the related enumerations for pin information.}

\textcolor{comment}{// Setup for GPIO 22 CE and CE0 CSN with SPI Speed @ 4Mhz}
\textcolor{comment}{//RF24 radio(RPI\_V2\_GPIO\_P1\_22, BCM2835\_SPI\_CS0, BCM2835\_SPI\_SPEED\_4MHZ);}

\textcolor{comment}{// NEW: Setup for RPi B+}
\textcolor{comment}{//RF24 radio(RPI\_BPLUS\_GPIO\_J8\_15,RPI\_BPLUS\_GPIO\_J8\_24, BCM2835\_SPI\_SPEED\_8MHZ);}

\textcolor{comment}{// Setup for GPIO 15 CE and CE0 CSN with SPI Speed @ 8Mhz}
\textcolor{comment}{//RF24 radio(RPI\_V2\_GPIO\_P1\_15, RPI\_V2\_GPIO\_P1\_24, BCM2835\_SPI\_SPEED\_8MHZ);}
\hyperlink{classRF24}{RF24} radio(25,8, BCM2835\_SPI\_SPEED\_8MHZ);

\textcolor{comment}{/*** RPi Alternate ***/}
\textcolor{comment}{//Note: Specify SPI BUS 0 or 1 instead of CS pin number.}
\textcolor{comment}{// See http://tmrh20.github.io/RF24/RPi.html for more information on usage}

\textcolor{comment}{//RPi Alternate, with MRAA}
\textcolor{comment}{//RF24 radio(15,0);}

\textcolor{comment}{//RPi Alternate, with SPIDEV - Note: Edit RF24/arch/BBB/spi.cpp and  set 'this->device =
       "/dev/spidev0.0";;' or as listed in /dev}
\textcolor{comment}{//RF24 radio(22,0);}


\textcolor{comment}{/****************** Linux (BBB,x86,etc) ***********************/}

\textcolor{comment}{// See http://tmrh20.github.io/RF24/pages.html for more information on usage}
\textcolor{comment}{// See http://iotdk.intel.com/docs/master/mraa/ for more information on MRAA}
\textcolor{comment}{// See https://www.kernel.org/doc/Documentation/spi/spidev for more information on SPIDEV}

\textcolor{comment}{// Setup for ARM(Linux) devices like BBB using spidev (default is "/dev/spidev1.0" )}
\textcolor{comment}{//RF24 radio(115,0);}

\textcolor{comment}{//BBB Alternate, with mraa}
\textcolor{comment}{// CE pin = (Header P9, Pin 13) = 59 = 13 + 46 }
\textcolor{comment}{//Note: Specify SPI BUS 0 or 1 instead of CS pin number. }
\textcolor{comment}{//RF24 radio(59,0);}

\textcolor{comment}{/**************************************************************/}

\textcolor{comment}{// Radio pipe addresses for the 2 nodes to communicate.}
\textcolor{keyword}{const} uint64\_t addresses[2] = \{ 0xABCDABCD71LL, 0x544d52687CLL \};


uint8\_t data[32];
\textcolor{keywordtype}{unsigned} \textcolor{keywordtype}{long} startTime, stopTime, counter, rxTimer=0;

\textcolor{keywordtype}{int} main(\textcolor{keywordtype}{int} argc, \textcolor{keywordtype}{char}** argv)\{

  \textcolor{keywordtype}{bool} role\_ping\_out = 1, role\_pong\_back = 0;
  \textcolor{keywordtype}{bool} role = 0;

  \textcolor{comment}{// Print preamble:}

  cout << \textcolor{stringliteral}{"RF24/examples/Transfer/\(\backslash\)n"};

  radio.begin();                           \textcolor{comment}{// Setup and configure rf radio}
  radio.setChannel(1);
  radio.setPALevel(\hyperlink{RF24_8h_a1e4cd0bea93e6b43422855fb0120aaceab0bfc94c4095e9495b2e49530b623d0d}{RF24\_PA\_MAX});
  radio.setDataRate(\hyperlink{RF24_8h_a82745de4aa1251b7561564b3ed1d6522ad6a241689903e120c99b6963cb98c97c}{RF24\_250KBPS});
  radio.setAutoAck(1);                     \textcolor{comment}{// Ensure autoACK is enabled}
  radio.setRetries(2,15);                  \textcolor{comment}{// Optionally, increase the delay between retries & # of retries}
  radio.setCRCLength(\hyperlink{RF24_8h_adbe00719f3f835c82bd007081d040a7eade0b6b3a0dd8729e2a17c49896e0a468}{RF24\_CRC\_8});          \textcolor{comment}{// Use 8-bit CRC for performance}
  radio.printDetails();
\textcolor{comment}{/********* Role chooser ***********/}

  printf(\textcolor{stringliteral}{"\(\backslash\)n ************ Role Setup ***********\(\backslash\)n"});
  \textcolor{keywordtype}{string} input = \textcolor{stringliteral}{""};
  \textcolor{keywordtype}{char} myChar = \{0\};
  cout << \textcolor{stringliteral}{"Choose a role: Enter 0 for receiver, 1 for transmitter (CTRL+C to exit)\(\backslash\)n>"};
  getline(cin,input);

  \textcolor{keywordflow}{if}(input.length() == 1) \{
    myChar = input[0];
    \textcolor{keywordflow}{if}(myChar == \textcolor{charliteral}{'0'})\{
        cout << \textcolor{stringliteral}{"Role: Pong Back, awaiting transmission "} << endl << endl;
    \}\textcolor{keywordflow}{else}\{  cout << \textcolor{stringliteral}{"Role: Ping Out, starting transmission "} << endl << endl;
        role = role\_ping\_out;
    \}
  \}
\textcolor{comment}{/***********************************/}

    \textcolor{keywordflow}{if} ( role == role\_ping\_out )    \{
      radio.openWritingPipe(addresses[1]);
      radio.openReadingPipe(1,addresses[0]);
      radio.stopListening();
    \} \textcolor{keywordflow}{else} \{
      radio.openWritingPipe(addresses[0]);
      radio.openReadingPipe(1,addresses[1]);
      radio.startListening();
    \}


  \textcolor{keywordflow}{for}(\textcolor{keywordtype}{int} i=0; i<32; i++)\{
     data[i] = rand() % 255;                        \textcolor{comment}{//Load the buffer with random data}
  \}

    \textcolor{comment}{// forever loop}
    \textcolor{keywordflow}{while} (1)\{

    \textcolor{keywordflow}{if} (role == role\_ping\_out)\{
        sleep(2);
        printf(\textcolor{stringliteral}{"Initiating Basic Data Transfer\(\backslash\)n\(\backslash\)r"});

        \textcolor{keywordtype}{long} \textcolor{keywordtype}{int} cycles = 10000;                    \textcolor{comment}{//Change this to a higher or lower number.}
        
        \textcolor{comment}{// unsigned long pauseTime = millis();      //Uncomment if autoAck == 1 ( NOACK )}
        startTime = \hyperlink{group__Porting__General_gad5b3ec1ce839fa1c4337a7d0312e9749}{millis}();
    
        \textcolor{keywordflow}{for}(\textcolor{keywordtype}{int} i=0; i<cycles; i++)\{                \textcolor{comment}{//Loop through a number of cycles}
                data[0] = i;                        \textcolor{comment}{//Change the first byte of the payload for
       identification}
                \textcolor{keywordflow}{if}(!radio.writeFast(&data,32))\{     \textcolor{comment}{//Write to the FIFO buffers}
                    counter++;                      \textcolor{comment}{//Keep count of failed payloads}
                \}

                
                \textcolor{comment}{//This is only required when NO ACK ( enableAutoAck(0) ) payloads are used}
        \textcolor{comment}{/*      if(millis() - pauseTime > 3)\{       // Need to drop out of TX mode every 4ms if sending a
       steady stream of multicast data}
\textcolor{comment}{                    pauseTime = millis();           }
\textcolor{comment}{                    radio.txStandBy();              // This gives the PLL time to sync back up  }
\textcolor{comment}{                \}}
\textcolor{comment}{        */}
        \}
        stopTime = \hyperlink{group__Porting__General_gad5b3ec1ce839fa1c4337a7d0312e9749}{millis}();

        \textcolor{keywordflow}{if}(!radio.txStandBy())\{ counter+=3; \}

        \textcolor{keywordtype}{float} numBytes = cycles*32;
        \textcolor{keywordtype}{float} rate = numBytes / (stopTime - startTime);

        printf(\textcolor{stringliteral}{"Transfer complete at %.2f KB/s \(\backslash\)n\(\backslash\)r"},rate);
        printf(\textcolor{stringliteral}{"%lu of %lu Packets Failed to Send\(\backslash\)n\(\backslash\)r"},counter,cycles);
        counter = 0;

    \}


\textcolor{keywordflow}{if}(role == role\_pong\_back)\{
     \textcolor{keywordflow}{while}(radio.available())\{
      radio.read(&data,32);
      counter++;
     \}
   \textcolor{keywordflow}{if}(\hyperlink{group__Porting__General_gad5b3ec1ce839fa1c4337a7d0312e9749}{millis}() - rxTimer > 1000)\{
     rxTimer = \hyperlink{group__Porting__General_gad5b3ec1ce839fa1c4337a7d0312e9749}{millis}();
     printf(\textcolor{stringliteral}{"Rate: "});
     \textcolor{keywordtype}{float} numBytes = counter*32;
     printf(\textcolor{stringliteral}{"%.2f KB/s \(\backslash\)n\(\backslash\)r"},numBytes/1000);
     printf(\textcolor{stringliteral}{"Payload Count: %lu \(\backslash\)n\(\backslash\)r"}, counter);
     counter = 0;
   \}
  \}

\} \textcolor{comment}{// loop}
\} \textcolor{comment}{// main}






\end{DoxyCodeInclude}
 