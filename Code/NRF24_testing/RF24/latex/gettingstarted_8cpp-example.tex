\hypertarget{gettingstarted_8cpp-example}{}\section{gettingstarted.\+cpp}
{\bfseries For Linux}~\newline
 {\bfseries Updated\+: T\+M\+Rh20 2014 }~\newline


This is an example of how to use the \hyperlink{classRF24}{R\+F24} class to communicate on a basic level. Configure and write this sketch to two different nodes. Put one of the nodes into \textquotesingle{}transmit\textquotesingle{} mode by connecting with the serial monitor and ~\newline
 sending a \textquotesingle{}T\textquotesingle{}. The ping node sends the current time to the pong node, which responds by sending the value back. The ping node can then see how long the whole cycle took. ~\newline
 \begin{DoxyNote}{Note}
For a more efficient call-\/response scenario see the Getting\+Started\+\_\+\+Call\+Response.\+ino example.
\end{DoxyNote}

\begin{DoxyCodeInclude}
\textcolor{comment}{/*}
\textcolor{comment}{ Copyright (C) 2011 J. Coliz <maniacbug@ymail.com>}
\textcolor{comment}{}
\textcolor{comment}{ This program is free software; you can redistribute it and/or}
\textcolor{comment}{ modify it under the terms of the GNU General Public License}
\textcolor{comment}{ version 2 as published by the Free Software Foundation.}
\textcolor{comment}{}
\textcolor{comment}{ 03/17/2013 : Charles-Henri Hallard (http://hallard.me)}
\textcolor{comment}{              Modified to use with Arduipi board http://hallard.me/arduipi}
\textcolor{comment}{                          Changed to use modified bcm2835 and RF24 library}
\textcolor{comment}{TMRh20 2014 - Updated to work with optimized RF24 Arduino library}
\textcolor{comment}{}
\textcolor{comment}{ */}

\textcolor{preprocessor}{#include <cstdlib>}
\textcolor{preprocessor}{#include <iostream>}
\textcolor{preprocessor}{#include <sstream>}
\textcolor{preprocessor}{#include <string>}
\textcolor{preprocessor}{#include <unistd.h>}
\textcolor{preprocessor}{#include <\hyperlink{RF24_8h}{RF24/RF24.h}>}

\textcolor{keyword}{using namespace }\hyperlink{namespacestd}{std};
\textcolor{comment}{//}
\textcolor{comment}{// Hardware configuration}
\textcolor{comment}{// Configure the appropriate pins for your connections}

\textcolor{comment}{/****************** Raspberry Pi ***********************/}

\textcolor{comment}{// Radio CE Pin, CSN Pin, SPI Speed}
\textcolor{comment}{// See http://www.airspayce.com/mikem/bcm2835/group\_\_constants.html#ga63c029bd6500167152db4e57736d0939 and
       the related enumerations for pin information.}

\textcolor{comment}{// Setup for GPIO 22 CE and CE0 CSN with SPI Speed @ 4Mhz}
\textcolor{comment}{//RF24 radio(RPI\_V2\_GPIO\_P1\_22, BCM2835\_SPI\_CS0, BCM2835\_SPI\_SPEED\_4MHZ);}

\textcolor{comment}{// NEW: Setup for RPi B+}
\textcolor{comment}{//RF24 radio(RPI\_BPLUS\_GPIO\_J8\_15,RPI\_BPLUS\_GPIO\_J8\_24, BCM2835\_SPI\_SPEED\_8MHZ);}

\textcolor{comment}{// Setup for GPIO 15 CE and CE0 CSN with SPI Speed @ 8Mhz}
\textcolor{comment}{//RF24 radio(RPI\_V2\_GPIO\_P1\_15, RPI\_V2\_GPIO\_P1\_24, BCM2835\_SPI\_SPEED\_8MHZ);}
\hyperlink{classRF24}{RF24} radio(25,8, BCM2835\_SPI\_SPEED\_8MHZ);

\textcolor{comment}{// RPi generic:}
\textcolor{comment}{//RF24 radio(22,0);}

\textcolor{comment}{/*** RPi Alternate ***/}
\textcolor{comment}{//Note: Specify SPI BUS 0 or 1 instead of CS pin number.}
\textcolor{comment}{// See http://tmrh20.github.io/RF24/RPi.html for more information on usage}

\textcolor{comment}{//RPi Alternate, with MRAA}
\textcolor{comment}{//RF24 radio(15,0);}

\textcolor{comment}{//RPi Alternate, with SPIDEV - Note: Edit RF24/arch/BBB/spi.cpp and  set 'this->device =
       "/dev/spidev0.0";;' or as listed in /dev}
\textcolor{comment}{//RF24 radio(22,0);}


\textcolor{comment}{/****************** Linux (BBB,x86,etc) ***********************/}

\textcolor{comment}{// See http://tmrh20.github.io/RF24/pages.html for more information on usage}
\textcolor{comment}{// See http://iotdk.intel.com/docs/master/mraa/ for more information on MRAA}
\textcolor{comment}{// See https://www.kernel.org/doc/Documentation/spi/spidev for more information on SPIDEV}

\textcolor{comment}{// Setup for ARM(Linux) devices like BBB using spidev (default is "/dev/spidev1.0" )}
\textcolor{comment}{//RF24 radio(115,0);}

\textcolor{comment}{//BBB Alternate, with mraa}
\textcolor{comment}{// CE pin = (Header P9, Pin 13) = 59 = 13 + 46 }
\textcolor{comment}{//Note: Specify SPI BUS 0 or 1 instead of CS pin number. }
\textcolor{comment}{//RF24 radio(59,0);}

\textcolor{comment}{/********** User Config *********/}
\textcolor{comment}{// Assign a unique identifier for this node, 0 or 1}
\textcolor{keywordtype}{bool} radioNumber = 1;

\textcolor{comment}{/********************************/}

\textcolor{comment}{// Radio pipe addresses for the 2 nodes to communicate.}
\textcolor{keyword}{const} uint8\_t pipes[][6] = \{\textcolor{stringliteral}{"1Node"},\textcolor{stringliteral}{"2Node"}\};


\textcolor{keywordtype}{int} main(\textcolor{keywordtype}{int} argc, \textcolor{keywordtype}{char}** argv)\{

  \textcolor{keywordtype}{bool} role\_ping\_out = \textcolor{keyword}{true}, role\_pong\_back = \textcolor{keyword}{false};
  \textcolor{keywordtype}{bool} role = role\_pong\_back;

  cout << \textcolor{stringliteral}{"RF24/examples/GettingStarted/\(\backslash\)n"};

  \textcolor{comment}{// Setup and configure rf radio}
  radio.begin();

  \textcolor{comment}{// optionally, increase the delay between retries & # of retries}
  radio.setRetries(15,15);
  \textcolor{comment}{// Dump the configuration of the rf unit for debugging}
  radio.printDetails();


\textcolor{comment}{/********* Role chooser ***********/}

  printf(\textcolor{stringliteral}{"\(\backslash\)n ************ Role Setup ***********\(\backslash\)n"});
  \textcolor{keywordtype}{string} input = \textcolor{stringliteral}{""};
  \textcolor{keywordtype}{char} myChar = \{0\};
  cout << \textcolor{stringliteral}{"Choose a role: Enter 0 for pong\_back, 1 for ping\_out (CTRL+C to exit) \(\backslash\)n>"};
  getline(cin,input);

  \textcolor{keywordflow}{if}(input.length() == 1) \{
    myChar = input[0];
    \textcolor{keywordflow}{if}(myChar == \textcolor{charliteral}{'0'})\{
        cout << \textcolor{stringliteral}{"Role: Pong Back, awaiting transmission "} << endl << endl;
    \}\textcolor{keywordflow}{else}\{  cout << \textcolor{stringliteral}{"Role: Ping Out, starting transmission "} << endl << endl;
        role = role\_ping\_out;
    \}
  \}
\textcolor{comment}{/***********************************/}
  \textcolor{comment}{// This simple sketch opens two pipes for these two nodes to communicate}
  \textcolor{comment}{// back and forth.}

    \textcolor{keywordflow}{if} ( !radioNumber )    \{
      radio.openWritingPipe(pipes[0]);
      radio.openReadingPipe(1,pipes[1]);
    \} \textcolor{keywordflow}{else} \{
      radio.openWritingPipe(pipes[1]);
      radio.openReadingPipe(1,pipes[0]);
    \}
    
    radio.startListening();
    
    \textcolor{comment}{// forever loop}
    \textcolor{keywordflow}{while} (1)
    \{
        \textcolor{keywordflow}{if} (role == role\_ping\_out)
        \{
            \textcolor{comment}{// First, stop listening so we can talk.}
            radio.stopListening();

            \textcolor{comment}{// Take the time, and send it.  This will block until complete}

            printf(\textcolor{stringliteral}{"Now sending...\(\backslash\)n"});
            \textcolor{keywordtype}{unsigned} \textcolor{keywordtype}{long} time = \hyperlink{group__Porting__General_gad5b3ec1ce839fa1c4337a7d0312e9749}{millis}();

            \textcolor{keywordtype}{bool} ok = radio.write( &time, \textcolor{keyword}{sizeof}(\textcolor{keywordtype}{unsigned} \textcolor{keywordtype}{long}) );

            \textcolor{keywordflow}{if} (!ok)\{
                printf(\textcolor{stringliteral}{"failed.\(\backslash\)n"});
            \}
            \textcolor{comment}{// Now, continue listening}
            radio.startListening();

            \textcolor{comment}{// Wait here until we get a response, or timeout (250ms)}
            \textcolor{keywordtype}{unsigned} \textcolor{keywordtype}{long} started\_waiting\_at = \hyperlink{group__Porting__General_gad5b3ec1ce839fa1c4337a7d0312e9749}{millis}();
            \textcolor{keywordtype}{bool} timeout = \textcolor{keyword}{false};
            \textcolor{keywordflow}{while} ( ! radio.available() && ! timeout ) \{
                \textcolor{keywordflow}{if} (\hyperlink{group__Porting__General_gad5b3ec1ce839fa1c4337a7d0312e9749}{millis}() - started\_waiting\_at > 200 )
                    timeout = \textcolor{keyword}{true};
            \}


            \textcolor{comment}{// Describe the results}
            \textcolor{keywordflow}{if} ( timeout )
            \{
                printf(\textcolor{stringliteral}{"Failed, response timed out.\(\backslash\)n"});
            \}
            \textcolor{keywordflow}{else}
            \{
                \textcolor{comment}{// Grab the response, compare, and send to debugging spew}
                \textcolor{keywordtype}{unsigned} \textcolor{keywordtype}{long} got\_time;
                radio.read( &got\_time, \textcolor{keyword}{sizeof}(\textcolor{keywordtype}{unsigned} \textcolor{keywordtype}{long}) );

                \textcolor{comment}{// Spew it}
                printf(\textcolor{stringliteral}{"Got response %lu, round-trip delay: %lu\(\backslash\)n"},got\_time,
      \hyperlink{group__Porting__General_gad5b3ec1ce839fa1c4337a7d0312e9749}{millis}()-got\_time);
            \}
            sleep(1);
        \}

        \textcolor{comment}{//}
        \textcolor{comment}{// Pong back role.  Receive each packet, dump it out, and send it back}
        \textcolor{comment}{//}

        \textcolor{keywordflow}{if} ( role == role\_pong\_back )
        \{
            
            \textcolor{comment}{// if there is data ready}
            \textcolor{keywordflow}{if} ( radio.available() )
            \{
                \textcolor{comment}{// Dump the payloads until we've gotten everything}
                \textcolor{keywordtype}{unsigned} \textcolor{keywordtype}{long} got\_time;

                \textcolor{comment}{// Fetch the payload, and see if this was the last one.}
                \textcolor{keywordflow}{while}(radio.available())\{
                    radio.read( &got\_time, \textcolor{keyword}{sizeof}(\textcolor{keywordtype}{unsigned} \textcolor{keywordtype}{long}) );
                \}
                radio.stopListening();
                
                radio.write( &got\_time, \textcolor{keyword}{sizeof}(\textcolor{keywordtype}{unsigned} \textcolor{keywordtype}{long}) );

                \textcolor{comment}{// Now, resume listening so we catch the next packets.}
                radio.startListening();

                \textcolor{comment}{// Spew it}
                printf(\textcolor{stringliteral}{"Got payload(%d) %lu...\(\backslash\)n"},\textcolor{keyword}{sizeof}(\textcolor{keywordtype}{unsigned} \textcolor{keywordtype}{long}), got\_time);
                
                \hyperlink{group__Porting__General_ga70a331e8ddf9acf9d33c47b71cda4c5f}{delay}(925); \textcolor{comment}{//Delay after payload responded to, minimize RPi CPU time}
                
            \}
        
        \}

    \} \textcolor{comment}{// forever loop}

  \textcolor{keywordflow}{return} 0;
\}

\end{DoxyCodeInclude}
 