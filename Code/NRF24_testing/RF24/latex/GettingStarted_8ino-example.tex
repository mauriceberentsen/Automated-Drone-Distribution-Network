\hypertarget{GettingStarted_8ino-example}{}\section{Getting\+Started.\+ino}
{\bfseries For Arduino}~\newline
 {\bfseries Updated\+: T\+M\+Rh20 2014 }~\newline


This is an example of how to use the \hyperlink{classRF24}{R\+F24} class to communicate on a basic level. Configure and write this sketch to two different nodes. Put one of the nodes into \textquotesingle{}transmit\textquotesingle{} mode by connecting with the serial monitor and ~\newline
 sending a \textquotesingle{}T\textquotesingle{}. The ping node sends the current time to the pong node, which responds by sending the value back. The ping node can then see how long the whole cycle took. ~\newline
 \begin{DoxyNote}{Note}
For a more efficient call-\/response scenario see the Getting\+Started\+\_\+\+Call\+Response.\+ino example. 

When switching between sketches, the radio may need to be powered down to clear settings that are not \char`\"{}un-\/set\char`\"{} otherwise
\end{DoxyNote}

\begin{DoxyCodeInclude}

\textcolor{comment}{/*}
\textcolor{comment}{* Getting Started example sketch for nRF24L01+ radios}
\textcolor{comment}{* This is a very basic example of how to send data from one node to another}
\textcolor{comment}{* Updated: Dec 2014 by TMRh20}
\textcolor{comment}{*/}

\textcolor{preprocessor}{#include <SPI.h>}
\textcolor{preprocessor}{#include "\hyperlink{RF24_8h}{RF24.h}"}

\textcolor{comment}{/****************** User Config ***************************/}
\textcolor{comment}{/***      Set this radio as radio number 0 or 1         ***/}
\textcolor{keywordtype}{bool} radioNumber = 0;

\textcolor{comment}{/* Hardware configuration: Set up nRF24L01 radio on SPI bus plus pins 7 & 8 */}
\hyperlink{classRF24}{RF24} radio(7,8);
\textcolor{comment}{/**********************************************************/}

byte addresses[][6] = \{\textcolor{stringliteral}{"1Node"},\textcolor{stringliteral}{"2Node"}\};

\textcolor{comment}{// Used to control whether this node is sending or receiving}
\textcolor{keywordtype}{bool} role = 0;

\textcolor{keywordtype}{void} setup() \{
  Serial.begin(115200);
  Serial.println(F(\textcolor{stringliteral}{"RF24/examples/GettingStarted"}));
  Serial.println(F(\textcolor{stringliteral}{"*** PRESS 'T' to begin transmitting to the other node"}));
  
  radio.begin();

  \textcolor{comment}{// Set the PA Level low to prevent power supply related issues since this is a}
 \textcolor{comment}{// getting\_started sketch, and the likelihood of close proximity of the devices. RF24\_PA\_MAX is default.}
  radio.setPALevel(\hyperlink{RF24_8h_a1e4cd0bea93e6b43422855fb0120aacea7d8d09f4a047b7c22655e56c98ca010c}{RF24\_PA\_LOW});
  
  \textcolor{comment}{// Open a writing and reading pipe on each radio, with opposite addresses}
  \textcolor{keywordflow}{if}(radioNumber)\{
    radio.openWritingPipe(addresses[1]);
    radio.openReadingPipe(1,addresses[0]);
  \}\textcolor{keywordflow}{else}\{
    radio.openWritingPipe(addresses[0]);
    radio.openReadingPipe(1,addresses[1]);
  \}
  
  \textcolor{comment}{// Start the radio listening for data}
  radio.startListening();
\}

\textcolor{keywordtype}{void} loop() \{
  
  
\textcolor{comment}{/****************** Ping Out Role ***************************/}  
\textcolor{keywordflow}{if} (role == 1)  \{
    
    radio.stopListening();                                    \textcolor{comment}{// First, stop listening so we can talk.}
    
    
    Serial.println(F(\textcolor{stringliteral}{"Now sending"}));

    \textcolor{keywordtype}{unsigned} \textcolor{keywordtype}{long} start\_time = micros();                             \textcolor{comment}{// Take the time, and send it.  This
       will block until complete}
     \textcolor{keywordflow}{if} (!radio.write( &start\_time, \textcolor{keyword}{sizeof}(\textcolor{keywordtype}{unsigned} \textcolor{keywordtype}{long}) ))\{
       Serial.println(F(\textcolor{stringliteral}{"failed"}));
     \}
        
    radio.startListening();                                    \textcolor{comment}{// Now, continue listening}
    
    \textcolor{keywordtype}{unsigned} \textcolor{keywordtype}{long} started\_waiting\_at = micros();               \textcolor{comment}{// Set up a timeout period, get the current
       microseconds}
    \textcolor{keywordtype}{boolean} timeout = \textcolor{keyword}{false};                                   \textcolor{comment}{// Set up a variable to indicate if a
       response was received or not}
    
    \textcolor{keywordflow}{while} ( ! radio.available() )\{                             \textcolor{comment}{// While nothing is received}
      \textcolor{keywordflow}{if} (micros() - started\_waiting\_at > 200000 )\{            \textcolor{comment}{// If waited longer than 200ms, indicate
       timeout and exit while loop}
          timeout = \textcolor{keyword}{true};
          \textcolor{keywordflow}{break};
      \}      
    \}
        
    \textcolor{keywordflow}{if} ( timeout )\{                                             \textcolor{comment}{// Describe the results}
        Serial.println(F(\textcolor{stringliteral}{"Failed, response timed out."}));
    \}\textcolor{keywordflow}{else}\{
        \textcolor{keywordtype}{unsigned} \textcolor{keywordtype}{long} got\_time;                                 \textcolor{comment}{// Grab the response, compare, and send to
       debugging spew}
        radio.read( &got\_time, \textcolor{keyword}{sizeof}(\textcolor{keywordtype}{unsigned} \textcolor{keywordtype}{long}) );
        \textcolor{keywordtype}{unsigned} \textcolor{keywordtype}{long} end\_time = micros();
        
        \textcolor{comment}{// Spew it}
        Serial.print(F(\textcolor{stringliteral}{"Sent "}));
        Serial.print(start\_time);
        Serial.print(F(\textcolor{stringliteral}{", Got response "}));
        Serial.print(got\_time);
        Serial.print(F(\textcolor{stringliteral}{", Round-trip delay "}));
        Serial.print(end\_time-start\_time);
        Serial.println(F(\textcolor{stringliteral}{" microseconds"}));
    \}

    \textcolor{comment}{// Try again 1s later}
    \hyperlink{group__Porting__General_ga70a331e8ddf9acf9d33c47b71cda4c5f}{delay}(1000);
  \}



\textcolor{comment}{/****************** Pong Back Role ***************************/}

  \textcolor{keywordflow}{if} ( role == 0 )
  \{
    \textcolor{keywordtype}{unsigned} \textcolor{keywordtype}{long} got\_time;
    
    \textcolor{keywordflow}{if}( radio.available())\{
                                                                    \textcolor{comment}{// Variable for the received timestamp}
      \textcolor{keywordflow}{while} (radio.available()) \{                                   \textcolor{comment}{// While there is data ready}
        radio.read( &got\_time, \textcolor{keyword}{sizeof}(\textcolor{keywordtype}{unsigned} \textcolor{keywordtype}{long}) );             \textcolor{comment}{// Get the payload}
      \}
     
      radio.stopListening();                                        \textcolor{comment}{// First, stop listening so we can talk
         }
      radio.write( &got\_time, \textcolor{keyword}{sizeof}(\textcolor{keywordtype}{unsigned} \textcolor{keywordtype}{long}) );              \textcolor{comment}{// Send the final one back.      }
      radio.startListening();                                       \textcolor{comment}{// Now, resume listening so we catch
       the next packets.     }
      Serial.print(F(\textcolor{stringliteral}{"Sent response "}));
      Serial.println(got\_time);  
   \}
 \}




\textcolor{comment}{/****************** Change Roles via Serial Commands ***************************/}

  \textcolor{keywordflow}{if} ( Serial.available() )
  \{
    \textcolor{keywordtype}{char} c = toupper(Serial.read());
    \textcolor{keywordflow}{if} ( c == \textcolor{charliteral}{'T'} && role == 0 )\{      
      Serial.println(F(\textcolor{stringliteral}{"*** CHANGING TO TRANSMIT ROLE -- PRESS 'R' TO SWITCH BACK"}));
      role = 1;                  \textcolor{comment}{// Become the primary transmitter (ping out)}
    
   \}\textcolor{keywordflow}{else}
    \textcolor{keywordflow}{if} ( c == \textcolor{charliteral}{'R'} && role == 1 )\{
      Serial.println(F(\textcolor{stringliteral}{"*** CHANGING TO RECEIVE ROLE -- PRESS 'T' TO SWITCH BACK"}));      
       role = 0;                \textcolor{comment}{// Become the primary receiver (pong back)}
       radio.startListening();
       
    \}
  \}


\} \textcolor{comment}{// Loop}

\end{DoxyCodeInclude}
 