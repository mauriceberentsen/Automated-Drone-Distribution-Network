\hyperlink{classRF24}{R\+F24} supports a variety of Linux based devices via various drivers. Some boards like R\+Pi can utilize multiple methods to drive the G\+P\+IO and S\+PI functionality.

~\newline
 \hypertarget{RPi_PreConfig}{}\section{Potential Pre\+Configuration}\label{RPi_PreConfig}
If S\+PI is not already enabled, load it on boot\+: 
\begin{DoxyCode}
sudo raspi-config  
\end{DoxyCode}
 A. Update the tool via the menu as required~\newline
 B. Select {\bfseries Advanced} and {\bfseries enable the S\+PI kernel module} ~\newline
 C. Update other software and libraries 
\begin{DoxyCode}
sudo apt-\textcolor{keyword}{get} update 
\end{DoxyCode}
 
\begin{DoxyCode}
sudo apt-\textcolor{keyword}{get} upgrade 
\end{DoxyCode}
 ~\newline
~\newline
\hypertarget{RPi_Build}{}\section{Build Options}\label{RPi_Build}
The default build on Raspberry Pi utilizes the included {\bfseries B\+C\+M2835} driver from \href{http://www.airspayce.com/mikem/bcm2835}{\tt http\+://www.\+airspayce.\+com/mikem/bcm2835}
\begin{DoxyEnumerate}
\item 
\begin{DoxyCode}
sudo make install -B 
\end{DoxyCode}

\end{DoxyEnumerate}

Build using the {\bfseries M\+R\+AA} library from \href{http://iotdk.intel.com/docs/master/mraa/index.html}{\tt http\+://iotdk.\+intel.\+com/docs/master/mraa/index.\+html} ~\newline
 M\+R\+AA is not included. See the \href{MRAA.html}{\tt M\+R\+AA} platform page for more information.


\begin{DoxyEnumerate}
\item Install, and build M\+R\+AA 
\begin{DoxyCode}
git clone https:\textcolor{comment}{//github.com/intel-iot-devkit/mraa.git}
cd mraa
mkdir build
cd build
cmake .. -DBUILDSWIGNODE=OFF
sudo make install
\end{DoxyCode}

\item Complete the install ~\newline
 
\begin{DoxyCode}
nano /etc/ld.so.conf 
\end{DoxyCode}
 Add the line
\begin{DoxyCode}
/usr/local/lib/arm-linux-gnueabihf 
\end{DoxyCode}
 Run
\begin{DoxyCode}
sudo ldconfig 
\end{DoxyCode}

\item Install \hyperlink{classRF24}{R\+F24}, using M\+R\+AA 
\begin{DoxyCode}
./configure --driver=MRAA
sudo make install -B
\end{DoxyCode}
 See the gettingstarted example for an example of pin configuration
\end{DoxyEnumerate}

Build using {\bfseries S\+P\+I\+D\+EV}


\begin{DoxyEnumerate}
\item Make sure that spi device support is enabled and /dev/spidev$<$a$>$.$<$b$>$ is present
\item Install \hyperlink{classRF24}{R\+F24}, using S\+P\+I\+D\+EV 
\begin{DoxyCode}
./configure --driver=SPIDEV
sudo make install -B
\end{DoxyCode}

\item See the gettingstarted example for an example of pin configuration
\end{DoxyEnumerate}

~\newline
 \hypertarget{RPi_Pins}{}\section{Connections and Pin Configuration}\label{RPi_Pins}
Using pin 15/\+G\+P\+IO 22 for CE, pin 24/\+G\+P\+I\+O8 (C\+E0) for C\+SN

Can use either R\+Pi C\+E0 or C\+E1 pins for radio C\+SN.~\newline
 Choose any R\+Pi output pin for radio CE pin.

{\bfseries B\+C\+M2835 Constructor\+:} 
\begin{DoxyCode}
\hyperlink{classRF24}{RF24} radio(RPI\_V2\_GPIO\_P1\_15,BCM2835\_SPI\_CS0, BCM2835\_SPI\_SPEED\_8MHZ);
 or
\hyperlink{classRF24}{RF24} radio(RPI\_V2\_GPIO\_P1\_15,BCM2835\_SPI\_CS1, BCM2835\_SPI\_SPEED\_8MHZ);
  
RPi B+:
\hyperlink{classRF24}{RF24} radio(RPI\_BPLUS\_GPIO\_J8\_15,RPI\_BPLUS\_GPIO\_J8\_24, BCM2835\_SPI\_SPEED\_8MHZ);
or
\hyperlink{classRF24}{RF24} radio(RPI\_BPLUS\_GPIO\_J8\_15,RPI\_BPLUS\_GPIO\_J8\_26, BCM2835\_SPI\_SPEED\_8MHZ);

General:
\hyperlink{classRF24}{RF24} radio(22,0);
or
\hyperlink{classRF24}{RF24} radio(22,1);
\end{DoxyCode}
 See the gettingstarted example for an example of pin configuration

See \href{http://www.airspayce.com/mikem/bcm2835/index.html}{\tt http\+://www.\+airspayce.\+com/mikem/bcm2835/index.\+html} for B\+C\+M2835 class documentation. ~\newline
~\newline
 {\bfseries M\+R\+AA Constructor\+:}


\begin{DoxyCode}
\hyperlink{classRF24}{RF24} radio(15,0); 
\end{DoxyCode}


See \href{http://iotdk.intel.com/docs/master/mraa/rasppi.html}{\tt http\+://iotdk.\+intel.\+com/docs/master/mraa/rasppi.\+html} ~\newline
~\newline
 {\bfseries S\+P\+I\+\_\+\+D\+EV Constructor}


\begin{DoxyCode}
\hyperlink{classRF24}{RF24} radio(22,0); 
\end{DoxyCode}
 In general, use
\begin{DoxyCode}
\hyperlink{classRF24}{RF24} radio(<ce\_pin>, <a>*10+<b>); 
\end{DoxyCode}
 for proper S\+P\+I\+D\+EV constructor to address correct spi device at /dev/spidev$<$a$>$.$<$b$>$

See \href{http://pi.gadgetoid.com/pinout}{\tt http\+://pi.\+gadgetoid.\+com/pinout}

{\bfseries Pins\+:}

\tabulinesep=1mm
\begin{longtabu} spread 0pt [c]{*{4}{|X[-1]}|}
\hline
\rowcolor{\tableheadbgcolor}\textbf{ P\+IN }&\textbf{ N\+R\+F24\+L01 }&\textbf{ R\+PI }&\textbf{ R\+Pi -\/\+P1 Connector  }\\\cline{1-4}
\endfirsthead
\hline
\endfoot
\hline
\rowcolor{\tableheadbgcolor}\textbf{ P\+IN }&\textbf{ N\+R\+F24\+L01 }&\textbf{ R\+PI }&\textbf{ R\+Pi -\/\+P1 Connector  }\\\cline{1-4}
\endhead
1 &G\+ND &rpi-\/gnd &(25) \\\cline{1-4}
2 &V\+CC &rpi-\/3v3 &(17) \\\cline{1-4}
3 &CE &rpi-\/gpio22 &(15) \\\cline{1-4}
4 &C\+SN &rpi-\/gpio8 &(24) \\\cline{1-4}
5 &S\+CK &rpi-\/sckl &(23) \\\cline{1-4}
6 &M\+O\+SI &rpi-\/mosi &(19) \\\cline{1-4}
7 &M\+I\+SO &rpi-\/miso &(21) \\\cline{1-4}
8 &I\+RQ &-\/ &-\/ \\\cline{1-4}
\end{longtabu}


~\newline
~\newline


Based on the arduino lib from J. Coliz \href{mailto:maniacbug@ymail.com}{\tt maniacbug@ymail.\+com} ~\newline
 the library was berryfied by Purinda Gunasekara \href{mailto:purinda@gmail.com}{\tt purinda@gmail.\+com} ~\newline
 then forked from github stanleyseow/\+R\+F24 to \href{https://github.com/jscrane/RF24-rpi}{\tt https\+://github.\+com/jscrane/\+R\+F24-\/rpi} ~\newline
 Network lib also based on \href{https://github.com/farconada/RF24Network}{\tt https\+://github.\+com/farconada/\+R\+F24\+Network}

~\newline
~\newline
~\newline
 