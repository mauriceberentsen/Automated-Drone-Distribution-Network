The \hyperlink{classRF24}{R\+F24} driver can be build as a static library with Atmel Studio 7 in order to be included as any other library in another program for the X\+M\+E\+GA family.

Currently only the {\bfseries A\+T\+X\+M\+E\+GA D3} family is implemented.\hypertarget{ATXMEGA_Preparation}{}\section{Preparation}\label{ATXMEGA_Preparation}
Create an empty G\+CC Static Library project in A\+S7.~\newline
 As not all files are required, copy the following directory structure in the project\+:


\begin{DoxyCode}
utility\(\backslash\)
  ATXMegaD3\(\backslash\)
    compatibility.c
    compatibility.h
    gpio.cpp
    gpio.h
    gpio\_helper.c
    gpio\_helper.h
    includes.h
    RF24\_arch\_config.h
    spi.cpp
    spi.h
nRF24L01.h
printf.h
\hyperlink{classRF24}{RF24}.cpp
\hyperlink{classRF24}{RF24}.h
RF24\_config.h
\end{DoxyCode}
\hypertarget{ATXMEGA_Usage}{}\section{Usage}\label{ATXMEGA_Usage}
Add the library to your project!~\newline
 In the file where the {\bfseries main()} is put the following in order to update the millisecond functionality\+:


\begin{DoxyCode}
ISR(TCE0\_OVF\_vect)
\{
   update\_milisec();
\}
\end{DoxyCode}


Declare the rf24 radio with {\bfseries \hyperlink{classRF24}{R\+F24} radio(\+X\+M\+E\+G\+A\+\_\+\+P\+O\+R\+T\+C\+\_\+\+P\+I\+N3, X\+M\+E\+G\+A\+\_\+\+S\+P\+I\+\_\+\+P\+O\+R\+T\+\_\+\+C);}

First parameter is the CE pin which can be any available pin on the uC.

Second parameter is the CS which can be on port C ({\bfseries X\+M\+E\+G\+A\+\_\+\+S\+P\+I\+\_\+\+P\+O\+R\+T\+\_\+C}) or on port D ({\bfseries X\+M\+E\+G\+A\+\_\+\+S\+P\+I\+\_\+\+P\+O\+R\+T\+\_\+D}).

Call the $\ast$$\ast$\+\_\+\+\_\+start\+\_\+timer()$\ast$$\ast$ to start the millisecond timer.

\begin{DoxyNote}{Note}
Note about the millisecond functionality\+:~\newline

\end{DoxyNote}
The millisecond functionality is based on the T\+C\+E0 so don\textquotesingle{}t use these pins as IO.~\newline
 The operating frequency of the uC is 32\+M\+Hz. If you have other frequency change the T\+C\+E0 registers appropriatly in function $\ast$$\ast$\+\_\+\+\_\+start\+\_\+timer()$\ast$$\ast$ in {\bfseries compatibility.\+c} file for your frequency. 