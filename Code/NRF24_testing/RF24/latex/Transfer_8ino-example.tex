\hypertarget{Transfer_8ino-example}{}\section{Transfer.\+ino}
{\bfseries For Arduino}~\newline
 This example demonstrates half-\/rate transfer using the F\+I\+FO buffers~\newline


It is an example of how to use the \hyperlink{classRF24}{R\+F24} class. Write this sketch to two different nodes. Put one of the nodes into \textquotesingle{}transmit\textquotesingle{} mode by connecting ~\newline
 with the serial monitor and sending a \textquotesingle{}T\textquotesingle{}. The data transfer will begin, with the receiver displaying the payload count. (32\+Byte Payloads) ~\newline



\begin{DoxyCodeInclude}
\textcolor{comment}{/*}
\textcolor{comment}{ TMRh20 2014}
\textcolor{comment}{}
\textcolor{comment}{ This program is free software; you can redistribute it and/or}
\textcolor{comment}{ modify it under the terms of the GNU General Public License}
\textcolor{comment}{ version 2 as published by the Free Software Foundation.}
\textcolor{comment}{ */}

\textcolor{comment}{/*}
\textcolor{comment}{ General Data Transfer Rate Test}
\textcolor{comment}{ }
\textcolor{comment}{ This example demonstrates basic data transfer functionality with the }
\textcolor{comment}{ updated library. This example will display the transfer rates acheived }
\textcolor{comment}{ using the slower form of high-speed transfer using blocking-writes.}
\textcolor{comment}{ */}


\textcolor{preprocessor}{#include <SPI.h>}
\textcolor{preprocessor}{#include "\hyperlink{RF24_8h}{RF24.h}"}

\textcolor{comment}{/*************  USER Configuration *****************************/}
                                           \textcolor{comment}{// Hardware configuration}
\hyperlink{classRF24}{RF24} radio(7,8);                           \textcolor{comment}{// Set up nRF24L01 radio on SPI bus plus pins 7 & 8}

\textcolor{comment}{/***************************************************************/}

\textcolor{keyword}{const} uint64\_t pipes[2] = \{ 0xABCDABCD71LL, 0x544d52687CLL \};   \textcolor{comment}{// Radio pipe addresses for the 2 nodes to
       communicate.}

byte data[32];                             \textcolor{comment}{//Data buffer for testing data transfer speeds}

\textcolor{keywordtype}{unsigned} \textcolor{keywordtype}{long} counter, rxTimer;            \textcolor{comment}{//Counter and timer for keeping track transfer info}
\textcolor{keywordtype}{unsigned} \textcolor{keywordtype}{long} startTime, stopTime;  
\textcolor{keywordtype}{bool} TX=1,RX=0,role=0;

\textcolor{keywordtype}{void} setup(\textcolor{keywordtype}{void}) \{

  Serial.begin(115200);

  radio.begin();                           \textcolor{comment}{// Setup and configure rf radio}
  radio.setChannel(1);
  radio.setPALevel(\hyperlink{RF24_8h_a1e4cd0bea93e6b43422855fb0120aaceab0bfc94c4095e9495b2e49530b623d0d}{RF24\_PA\_MAX});           \textcolor{comment}{// If you want to save power use "RF24\_PA\_MIN" but
       keep in mind that reduces the module's range}
  radio.setDataRate(\hyperlink{RF24_8h_a82745de4aa1251b7561564b3ed1d6522afd01f3fd55247a67c0bcfd459fe17fdf}{RF24\_1MBPS});
  radio.setAutoAck(1);                     \textcolor{comment}{// Ensure autoACK is enabled}
  radio.setRetries(2,15);                  \textcolor{comment}{// Optionally, increase the delay between retries & # of retries}
  
  radio.setCRCLength(\hyperlink{RF24_8h_adbe00719f3f835c82bd007081d040a7eade0b6b3a0dd8729e2a17c49896e0a468}{RF24\_CRC\_8});          \textcolor{comment}{// Use 8-bit CRC for performance}
  radio.openWritingPipe(pipes[0]);
  radio.openReadingPipe(1,pipes[1]);
  
  radio.startListening();                  \textcolor{comment}{// Start listening}
  radio.printDetails();                    \textcolor{comment}{// Dump the configuration of the rf unit for debugging}
  
  Serial.println(F(\textcolor{stringliteral}{"\(\backslash\)n\(\backslash\)rRF24/examples/Transfer/"}));
  Serial.println(F(\textcolor{stringliteral}{"*** PRESS 'T' to begin transmitting to the other node"}));
  
  randomSeed(analogRead(0));               \textcolor{comment}{//Seed for random number generation}
  
  \textcolor{keywordflow}{for}(\textcolor{keywordtype}{int} i = 0; i < 32; i++)\{
     data[i] = random(255);                \textcolor{comment}{//Load the buffer with random data}
  \}
  radio.powerUp();                         \textcolor{comment}{//Power up the radio}
\}

\textcolor{keywordtype}{void} loop(\textcolor{keywordtype}{void})\{


  \textcolor{keywordflow}{if}(role == TX)\{
    \hyperlink{group__Porting__General_ga70a331e8ddf9acf9d33c47b71cda4c5f}{delay}(2000);
    
    Serial.println(F(\textcolor{stringliteral}{"Initiating Basic Data Transfer"}));
    
    
    \textcolor{keywordtype}{unsigned} \textcolor{keywordtype}{long} cycles = 10000; \textcolor{comment}{//Change this to a higher or lower number. }
    
    startTime = \hyperlink{group__Porting__General_gad5b3ec1ce839fa1c4337a7d0312e9749}{millis}();
    \textcolor{keywordtype}{unsigned} \textcolor{keywordtype}{long} pauseTime = \hyperlink{group__Porting__General_gad5b3ec1ce839fa1c4337a7d0312e9749}{millis}();
            
    \textcolor{keywordflow}{for}(\textcolor{keywordtype}{int} i=0; i<cycles; i++)\{        \textcolor{comment}{//Loop through a number of cycles}
      data[0] = i;                      \textcolor{comment}{//Change the first byte of the payload for identification}
      \textcolor{keywordflow}{if}(!radio.writeFast(&data,32))\{   \textcolor{comment}{//Write to the FIFO buffers        }
        counter++;                      \textcolor{comment}{//Keep count of failed payloads}
      \}
      
      \textcolor{comment}{//This is only required when NO ACK ( enableAutoAck(0) ) payloads are used}
\textcolor{comment}{//      if(millis() - pauseTime > 3)\{}
\textcolor{comment}{//        pauseTime = millis();}
\textcolor{comment}{//        radio.txStandBy();          // Need to drop out of TX mode every 4ms if sending a steady stream
       of multicast data}
\textcolor{comment}{//        //delayMicroseconds(130);     // This gives the PLL time to sync back up   }
\textcolor{comment}{//      \}}
      
    \}
    
   stopTime = \hyperlink{group__Porting__General_gad5b3ec1ce839fa1c4337a7d0312e9749}{millis}();   
                                         \textcolor{comment}{//This should be called to wait for completion and put the radio
       in standby mode after transmission, returns 0 if data still in FIFO (timed out), 1 if success}
   \textcolor{keywordflow}{if}(!radio.txStandBy())\{ counter+=3; \} \textcolor{comment}{//Standby, block only until FIFO empty or auto-retry timeout.
       Flush TX FIFO if failed}
   \textcolor{comment}{//radio.txStandBy(1000);              //Standby, using extended timeout period of 1 second}
   
   \textcolor{keywordtype}{float} numBytes = cycles*32;
   \textcolor{keywordtype}{float} rate = numBytes / (stopTime - startTime);
    
   Serial.print(\textcolor{stringliteral}{"Transfer complete at "}); Serial.print(rate); Serial.println(\textcolor{stringliteral}{" KB/s"});
   Serial.print(counter); Serial.print(\textcolor{stringliteral}{" of "}); Serial.print(cycles); Serial.println(\textcolor{stringliteral}{" Packets Failed to
       Send"});
   counter = 0;   
    
   \}
  
  
  
\textcolor{keywordflow}{if}(role == RX)\{
     \textcolor{keywordflow}{while}(radio.available())\{       
      radio.read(&data,32);
      counter++;
     \}
   \textcolor{keywordflow}{if}(\hyperlink{group__Porting__General_gad5b3ec1ce839fa1c4337a7d0312e9749}{millis}() - rxTimer > 1000)\{
     rxTimer = \hyperlink{group__Porting__General_gad5b3ec1ce839fa1c4337a7d0312e9749}{millis}();     
     \textcolor{keywordtype}{unsigned} \textcolor{keywordtype}{long} numBytes = counter*32;
     Serial.print(F(\textcolor{stringliteral}{"Rate: "}));
     \textcolor{comment}{//Prevent dividing into 0, which will cause issues over a period of time}
     Serial.println(numBytes > 0 ? numBytes/1000.0:0);
     Serial.print(F(\textcolor{stringliteral}{"Payload Count: "}));
     Serial.println(counter);
     counter = 0;
   \}
  \}
  \textcolor{comment}{//}
  \textcolor{comment}{// Change roles}
  \textcolor{comment}{//}

  \textcolor{keywordflow}{if} ( Serial.available() )
  \{
    \textcolor{keywordtype}{char} c = toupper(Serial.read());
    \textcolor{keywordflow}{if} ( c == \textcolor{charliteral}{'T'} && role == RX )
    \{
      Serial.println(F(\textcolor{stringliteral}{"*** CHANGING TO TRANSMIT ROLE -- PRESS 'R' TO SWITCH BACK"}));
      radio.openWritingPipe(pipes[1]);
      radio.openReadingPipe(1,pipes[0]);
      radio.stopListening();
      role = TX;                  \textcolor{comment}{// Become the primary transmitter (ping out)}
    \}
    \textcolor{keywordflow}{else} \textcolor{keywordflow}{if} ( c == \textcolor{charliteral}{'R'} && role == TX )
    \{
      radio.openWritingPipe(pipes[0]);
      radio.openReadingPipe(1,pipes[1]); 
      radio.startListening();
      Serial.println(F(\textcolor{stringliteral}{"*** CHANGING TO RECEIVE ROLE -- PRESS 'T' TO SWITCH BACK"}));
      role = RX;                \textcolor{comment}{// Become the primary receiver (pong back)}
    \}
  \}
\}
\end{DoxyCodeInclude}
 