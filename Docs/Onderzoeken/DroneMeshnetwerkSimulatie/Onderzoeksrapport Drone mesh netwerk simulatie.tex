\documentclass[a4paper, 11pt, oneside]{report} 
\usepackage[utf8]{inputenc}
\usepackage[dutch]{babel}
\usepackage{amsmath}
\usepackage{amsfonts}
\usepackage{amssymb}
\usepackage{graphicx}
\usepackage{caption}
\usepackage[table,xcdraw]{xcolor}
\usepackage[toc,page]{appendix}
\usepackage{hyperref}
\usepackage{titlesec}
\usepackage{listings}
\usepackage{float}
\usepackage{tikz}
\usetikzlibrary{trees}
\usepackage{tikz-qtree}
\usepackage{graphicx}
\usepackage{fancyref}
\usepackage{wrapfig}
\usepackage{url}
\usepackage{pdflscape}
\usepackage{fancyvrb}
\usepackage{fancyhdr}
\graphicspath{ {Afbeeldingen/} }
\usepackage{subfig}
\usepackage{tabularx}
\usepackage{apacite}
\usepackage{longtable}
\usepackage{titlecaps}
%\usepackage[T1]{fontenc}
\usepackage{titlesec, blindtext, color}
\definecolor{gray75}{gray}{0.75}
\newcommand{\hsp}{\hspace{20pt}}
\usepackage{pdfpages}


\newcolumntype{L}[1]{>{\raggedright\arraybackslash}p{#1}}

\titleformat{\chapter}[hang]{\huge\bfseries}{\thechapter\hsp\textcolor{gray75}{|}\hsp}{0pt}{\Large\bfseries}


\def\sectionautorefname{Paragraaf}
\def\chapterautorefname{Hoofdstuk}
\def\tableautorefname{Tabel}
\DeclareRobustCommand{\VAN}[3]{#2} % set up for citation

%% Sets page size and margins 
\usepackage[a4paper,top=3cm,bottom=3cm,left=3cm,right=3cm,marginparwidth=1.75cm]{geometry}

\author{M.W.J. Berentsen}

\title{Onderzoeksrapport}
\usepackage{titling}

\newcommand{\subtitle}[8]{%
	\posttitle{%
		\par\end{center}
	\begin{center}\large#1\end{center}
	\vskip0.5em
	\begin{center}\large#2\end{center}
	\begin{center}\large#3\end{center}
	\begin{center}\large#4\end{center}
    \begin{center}\large#5\end{center}
    \begin{center}\large#6\end{center}
    \begin{center}\large#7\end{center}
    \begin{center}\large#8\end{center}
	\vskip0.5em}%
}

\subtitle{Drone mesh netwerk simulatie}{HAN Arnhem}{561399}{MWJ.Berentsen@student.han.nl}{Versie 1}{Alten Nederland B.V.}{Docent: J. Visch, MSc}{Assessor: ir. C.G.R. van Uffelen}

\setlength{\parindent}{0pt}
\setlength{\parskip}{5pt plus 2pt minus 1pt}



\hypersetup{colorlinks=true, urlcolor=red,citecolor=black,linkcolor=blue}  % Colours hyperlinks in blue, but this can be distracting if there are many links.
\setcounter{tocdepth}{2}



\begin{document}
\begin{figure}
\begin{center}\includegraphics[scale=0.1]{alten}\end{center}
\end{figure}
\maketitle

%\section*{Voorwoord}
%\addcontentsline{toc}{section}{\protect\numberline{}Voorwoord}
%\pagebreak

%Geschikt voor minimaal 50 nodes; Kan slecht of geen signaal nabootsen

\tableofcontents
\clearpage
%\section*{Begrippenlijst}

% Please add the following required packages to your document preamble:
% \usepackage[table,xcdraw]{xcolor}
% If you use beamer only pass "xcolor=table" option, i.e. \documentclass[xcolor=table]{beamer}
%\begin{table}[H]
%\centering

%\label{begrippen}
%\begin{tabular}{|l|l|}
%\hline
%\rowcolor[HTML]{C0C0C0}
%Term        & Omschrijving                                                         \\ \hline
%term        & Omschrijving                                                      	\\ \hline

%\end{tabular}
%\caption{Begrippenlijst}
%\end{table}

%\clearpage

%\section*{Samenvatting}
%\addcontentsline{toc}{section}{\protect\numberline{}Samenvatting}
%\pagebreak

\chapter{Samenvatting}
Optioneel een samenvatting van het onderzoek. Hier kunnen medestudenten snel inzicht krijgen in wat jij hebt onderzocht en wat je conclusie is.
\begin{itemize}
	\item Kunnen derden snel inzicht krijgen in jouw onderzoek?
	\item Staat de conclusie erin vermeld?
\end{itemize}

\chapter{Inleiding}
\label{chapter:inleiding}
De inleiding beschrijft:
\begin{itemize}
\item Waarvoor het onderzoek gedaan wordt;
\item Beschrijf waarom het onderzoek nu wordt uitgevoerd;
\item Het doel van het onderzoek.
\end{itemize}

\chapter{Hoofd- en deelvragen}
Hier worden de hoofd- en deelvragen genoemd en onderbouwd.

\begin{itemize}
\item Is de onderzoeksmethode beargumenteerd?
\item Staat hier wat je wel gaat onderzoeken en wat je niet gaat onderzoeken?
\item Beschrijf de invloed van jouw onderzoek op het project.
\end{itemize}

\chapter{Criteria}
Maak hier een lijst met criteria aan de hand van de hoofd- en deelvragen.
\begin{itemize}
	
\item Zijn de criteria duidelijk opgesteld?
\item Waar moet de oplossing aan voldoen, staat dit erin?
\end{itemize}

\begin{table}[H]
	\centering
	\begin{tabular}{|l|l|l|}
		\hline
		\rowcolor[HTML]{C0C0C0} 
		Naam & Beschrijving & Kosten \\ \hline
		&              &        \\ \hline
		&              &        \\ \hline
	\end{tabular}
	\caption{Opgestelde criteria}
	\label{tab:criteria}
\end{table}

\chapter{literatuur}
Zet hier de achterliggende theorie.
\begin{itemize}
\item Staat erin vermeld dat je de bieb hebt toegepast uit de methodekaart van de HAN?
\item Staan er oplossingsrichtingen vermeld die voldoen aan de criteria?
\end{itemize}


\chapter{Oplossingsrichtingen}
\begin{itemize}
\item Beschrijf je hoe jouw oplossingsrichting de hoofdvraag beantwoord?
\item Beschrijft de oplossingsrichting de mogelijkheden?
\item Staat hier "Veld" vermeldt van de methodekaart van de HAN?
\end{itemize}

\chapter{Experimenten}
\begin{itemize}
\item Beschrijft waarom dit experiment van belang is.
\item Voldoet de code van het experiment aan de opgestelde eisen?
\item Voldoet het programma aan de opgestelde QoS?
\item Is gedocumenteerd waarom het programma voldoet aan de opgestelde eisen?
\item Is gedocumenteerd waarom het programma voldoet aan de opgestelde QoS?
\end{itemize}

\chapter{Conclusie}
\label{chapter:conclusie}
Conclusie uit resultaten. Herhaal en beantwoord de hoofdvraag.
\begin{itemize}
\item Trek een conclusie uit de resultaten van de deelvragen.
\item Wordt de hoofdvraag beantwoord?
\item Wat beveel je aan?
\end{itemize}


\bibliographystyle{apacite}
\bibliography{bilbliography.bib}

\clearpage
\appendix

\end{document}