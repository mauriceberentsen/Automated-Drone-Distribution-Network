\documentclass[a4paper, 11pt, oneside]{article} 
\usepackage[utf8]{inputenc}
\usepackage[dutch]{babel}
\usepackage{amsmath}
\usepackage{amsfonts}
\usepackage{amssymb}
\usepackage{graphicx}
\usepackage{caption}
\usepackage[table,xcdraw]{xcolor}
\usepackage[toc,page]{appendix}
\usepackage{hyperref}
\usepackage{titlesec}
\usepackage{listings}
\usepackage{float}
\usepackage{tikz}
\usetikzlibrary{trees}
\usepackage{tikz-qtree}
\usepackage{graphicx}
\usepackage{fancyref}
\usepackage{wrapfig}
\usepackage{url}
\usepackage{pdflscape}
\usepackage{fancyvrb}
\graphicspath{ {Afbeeldingen/} }
\usepackage{subfig}
\usepackage{tabularx}
\usepackage{apacite}
\usepackage{longtable}
\usepackage{titlecaps}
%\usepackage[T1]{fontenc}
\usepackage{titlesec, blindtext, color}
\usepackage{censor}
\censorruledepth=-.2ex
\censorruleheight=.1ex

\usepackage[a4paper,top=3cm,bottom=3cm,left=3cm,right=3cm,marginparwidth=1.75cm]{geometry}

\setlength{\parindent}{0pt}
\setlength{\parskip}{5pt plus 2pt minus 1pt}

\author{M.W.J. Berentsen}

\title{Iteration Assessment template}

\begin{document}
	\maketitle
\section{Inleiding}	

{[}\textit{De inleiding van het Iteration Assessment zegt iets over het doel van dit document en waar het betrekking op heeft. Besteed, indien van toepassing, ook aandacht aan het hoofddoel van deze iteratie}{]}.

Deze Iteration Assessment heeft betrekking op iteratie \xblackout{XX} van het project automated mesh drone network simulation van Alten Nederland B.V.. Doel van de assessment sessie is te kijken naar wat er in deze iteratie is gebeurd, wat er is bereikt, wat er niet is bereikt en waarom niet, en we ervan kunnen leren. De beoordeling van de iteratie leidt tot een kwalificatie ervan.

\section{Deelnemers}
{[}\textit{Geef weer welke personen deelnemen aan de assessment sessie.}{]}

\section{Belangrijke mijlpalen}
{[}\textit{Geef hier een gedetailleerd overzicht van de in deze iteratie geplande mijlpalen.}{]}

\begin{table}[H]
	\centering
	\begin{tabular}{|l|l|l|}
		\hline
		\rowcolor[HTML]{C0C0C0} 
		Mijlpaal                                                                                                                                                             & \begin{tabular}[c]{@{}l@{}}Behaalde\\ datum\end{tabular} & Eigenaar                                                                      \\ \hline
		\begin{tabular}[c]{@{}l@{}}{[}Denk aan af te ronden werkproducten, start van belangrijke\\  activiteiten zoals testen, demomomenten en opleveringen.{]}\end{tabular} & {[}deadline{]}                                           & \begin{tabular}[c]{@{}l@{}}{[}Indicatie\\ ja of nee{]}\end{tabular} \\ \hline
		Start iteratie                                                                                                                                                       &                                                          &                                                                               \\ \hline
		{[}Keur oplevering iteratie \xblackout{E1} goed{]}                                                                                                                               &                                                          &                                                                               \\ \hline
		{[}Keur specificatie Use Case \xblackout{12} goed{]}                                                                                                                              &                                                          &                                                                               \\ \hline
		{[}Oplevering product \xblackout{XXXXXXXX}{]}
		&                                                          &                                                                               \\ \hline
		Eind iteratie (evaluatieoverleg)  & &                                                          \\ \hline
	\end{tabular}
\end{table}

\section{Iteratiedoelen}
\textit{
[Geef hier een gedetailleerd overzicht van de binnen deze iteratie te halen doelen en te verrichten taken en welke persoon voor de realisatie ervan verantwoordelijk is.]}

\begin{table}[H]
	\centering
	\begin{tabular}{|l|l|l|}
		\hline
		\rowcolor[HTML]{C0C0C0} 
		Doel/taak                                                                                                                                                                     & Prioriteit                                                        & Behaald                                                                      \\ \hline
		\begin{tabular}[c]{@{}l@{}}{[}Denk aan maken of verder uitwerken van\\  werkproducten,maatregelen tegen risico’s\\  en het uitvoeren van ondersteunende taken{]}\end{tabular} & \begin{tabular}[c]{@{}l@{}}{[}MoSCoW\\ prioriteit{]}\end{tabular} & \begin{tabular}[c]{@{}l@{}}{[}Indicatie\\ ja of nee{]}\end{tabular} \\ \hline
		\begin{tabular}[c]{@{}l@{}}{[}Bouw Use Case 1: Autorisatie.\\ Alleen Basisscenario en Scenario 2.{]}\end{tabular}                                                             & {[}M{]}                                                           &  \\ \hline
		{[}Bouw Use Case 1: Autorisatie. Scenario 3.{]}                                                                                                                               & {[}S{]}                                                           & \\ \hline
		{[}Keur specificatie Use Case 2 goed{]}                                                                                                                                       &                                                                   &                                                                               \\ \hline
		{[}Aanvullen{]}                                                                                                                                                               &                                                                   &                                                                               \\ \hline
		Plan de volgende iteratie                                                                                                                                                     & M                                                                 &        \\ \hline
	\end{tabular}
\end{table}

\section{Wat ging er deze iteratie goed}
{[}\textit{Geef hier puntsgewijs de zaken die in positieve zin vermeldenswaard zijn. Denk daarbij ook aan de aanpak en best practices.}{]}

\section{Wat is voor verbetering vatbaar}
{[}\textit{Geef hier puntsgewijs de zaken die voor verbetering in aanmerking komen. Documenteer de oorzaak en bepaal te nemen tegenmaatregelen of stel ze bij.}{]}

\section{Actielijst}
{[}\textit{Geef hier de afgesproken acties voor de volgende iteratie(s) aan.}{]}

\section{Kwalificatie van de iteratie}
{[}\textit{Geef hier de kwalificatie van deze iteratie: uitzonderlijk, gehaald, gehaald met risico, onvoltooid, mislukt, gestopt. Deze kwalificaties hebben de volgende betekenis:}{]}
\begin{itemize}
\item \textbf{Uitzonderlijk: }de iteratie is vlekkeloos verlopen. Alle milestones en doelen zijn gehaald.
\item \textbf{Gehaald:} alle Must have milestones en doelen zijn gehaald.
\item \textbf{Gehaald met risico:} Bijna alle Must have milestones en doelen zijn gehaald en het is mogelijk om door te gaan zonder dat er extra iteraties nodig zijn.
\item \textbf{Onvoltooid:} Er zijn niet genoeg Must have milestones en doelen gehaald maar de juiste dingen zijn gedaan gedurende de iteratie.
\item \textbf{Mislukt:} Er zijn weinig Must have milestones en doelen gehaald; bijsturing noodzakelijk.
\item \textbf{Gestopt:} De iteratie is stopgezet.
 
\end{itemize}


\end{document}