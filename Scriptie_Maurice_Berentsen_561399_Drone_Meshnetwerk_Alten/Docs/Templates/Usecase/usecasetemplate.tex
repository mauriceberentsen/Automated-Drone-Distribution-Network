\documentclass[a4paper, 11pt, oneside]{article} 
\usepackage[utf8]{inputenc}
\usepackage[dutch]{babel}
\usepackage{amsmath}
\usepackage{amsfonts}
\usepackage{amssymb}
\usepackage{graphicx}
\usepackage{caption}
\usepackage[table,xcdraw]{xcolor}
\usepackage[toc,page]{appendix}
\usepackage{hyperref}
\usepackage{titlesec}
\usepackage{listings}
\usepackage{float}
\usepackage{tikz}
\usetikzlibrary{trees}
\usepackage{tikz-qtree}
\usepackage{graphicx}
\usepackage{fancyref}
\usepackage{wrapfig}
\usepackage{url}
\usepackage{pdflscape}
\usepackage{fancyvrb}
\graphicspath{ {Afbeeldingen/} }
\usepackage{subfig}
\usepackage{tabularx}
\usepackage{apacite}
\usepackage{longtable}
\usepackage{titlecaps}
%\usepackage[T1]{fontenc}
\usepackage{titlesec, blindtext, color}
\usepackage{censor}
\censorruledepth=-.2ex
\censorruleheight=.1ex

\usepackage[a4paper,top=3cm,bottom=3cm,left=3cm,right=3cm,marginparwidth=1.75cm]{geometry}

\setlength{\parindent}{0pt}
\setlength{\parskip}{5pt plus 2pt minus 1pt}

\author{M.W.J. Berentsen}

\title{Usecase template}

\begin{document}
	\maketitle
%\section{Use Case \xblackout{Naam van de usecase}}	

\section{Fully-dressed use case description}

\begin{table}[H]
	\centering
	\begin{tabular}{|l|}
		\hline
		Use Case: \textless{}use case naam\textgreater{}                                                                              \\ \hline
		Purpose: \textless{}doel use case\textgreater{}                                                                               \\ \hline
		\begin{tabular}[c]{@{}l@{}}Description of use case:  \\ \textless{}wat kan je met de use case doen\textgreater{}\end{tabular} \\ \hline
		Primary actor:  \textless{}actor\textgreater{}                                                                                \\ \hline
		Preconditions: \textless{}pre conditie\textgreater{}                                                                          \\ \hline
		Postconditions: \textless{}post conditie\textgreater{}                                                                        \\ \hline
	\end{tabular}
\end{table}

\section{Basic Flow (Main Success Scenario)}
{[}\textit{Vul deze tabel in met de happy flow van het systeem}{]}

\begin{table}[H]
	\centering
	\begin{tabular}{|l|l|}
		\hline
		\rowcolor[HTML]{C0C0C0} 
		Actor action  & System responsibility   \\ \hline
		Actie actor 1 & systeem repsonse /actie \\ \hline
		&                         \\ \hline
		&                         \\ \hline
		&                         \\ \hline
		&                         \\ \hline
	\end{tabular}
\end{table}

\section{Alternative Flows}
{[}\textit{Geef hier een de alternative flow weer van het systeem. Per alternatieve flow een tabel}{]}

\begin{table}[H]
	\centering
	\begin{tabular}{|l|l|}
		\hline
		\rowcolor[HTML]{C0C0C0} 
		Actor action  & System responsibility   \\ \hline
		Actie actor 1 & systeem repsonse /actie \\ \hline
		&                         \\ \hline
		&                         \\ \hline
		&                         \\ \hline
		&                         \\ \hline
	\end{tabular}
\end{table}



\end{document}